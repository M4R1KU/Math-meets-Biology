\documentclass{article}
\usepackage{amsmath}

\title{Math meets Biology}
\author{Mario Kunz, Xaver Hanushevsky}
\date{March 2023}

\begin{document}

\maketitle

\newpage

System zweier Differentialgleichungen zum Beschreiben der positiven Autoregulation:

\begin{align*}
    \frac{d[RNA]}{dt}&=v_{max}\cdot\frac{[P]}{K+[P]}-k_{dr}\cdot[RNA] \\
    \frac{d[P]}{dt}&=k_s\cdot[RNA]-k_{dp}\cdot[P]
\end{align*}

\begin{tabular}{l l}
     $[RNA]$ & Konzentration der transkribierten mRNA \\
     $[P]$ & Konzentration des Proteins/Transkriptionsfaktor \\
     $v_{max}$ & maximale Geschwindigkeit der Transkription; entspricht $k_t\cdot [D_T]$ \\
     $k_t$ & Geschwindigkeitskonstante der Transkription \\
     $[D_T]$ & Konzentration der DNA bzw. TF-Bindungsstelle \\
     $K$ & Gleichgewichtskonstante der Bindung vom TF an die DNA-Bindungsstelle \\
     $k_{dr}$ & Geschwindigkeitskonstante der mRNA-Degradatioon \\
     $k_{dp}$ & Geschwindigkeitskonstante der Proteindegradation
\end{tabular}

\newpage
\section{Values scraped from places}
Lac-Operon (https://doi.org/10.1109/embc46164.2021.9630940):\\
\begin{itemize}
    \item $A \cdot e^{-E/RT} = k$ = 853 L/mol*min at 37°C
    \item transcription \& translation rate = 0.4 min$^{-1}$
    \item protein decay rate of U = 0.056 min$^{-1}$
    \item mRNA decay rate of V = 0.36 min$^{-1}$
\end{itemize}




\end{document}
