\documentclass{article}
\usepackage{amsmath}

\title{Presentation Outline}
\author{Mario Kunz, Xaver Hanushevsky}
\date{March 2023}

\begin{document}

Folgende Punkte sind noch in keiner bestimmten Reihenfolge

\begin{itemize}
    \item DGL vorstellen für mind. negative oder positive Autoregulation
    \begin{itemize}
        \item Konstanten
        \item Zusammenhänge
        \item kinetischer Hintergrund
    \end{itemize}
    \item biologische Funktionsweise eines betrachteten Gens kurz erläutern
    \begin{itemize}
        \item Promoter
        \item Operator
    \end{itemize}
    \item Was nicht beachtet wurde
    \begin{itemize}
        \item Erweiterungsmöglichkeiten des Modells
    \end{itemize}
    \item Numerische Lösung der DGL
    \begin{itemize}
        \item Wie wurde die numerische Lösung erstellt?
        \item Gewählte Werte der Konstanten (mit Begründung)
        \item biologische Implikationen
    \end{itemize}
    \item stationäre Lösung und ihre Bedeutung
\end{itemize}

\section{Ideen}
\begin{itemize}
    \item Pathologische Werte (numerische Lösung)
    \item lokale Maxima voraussagen (nur für negative Autoregulation)
    \item 
\end{itemize}

\end{document}
